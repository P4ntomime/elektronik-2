\begin{center}
    \scalebox{0.5}{%
        \begin{circuitikz}[thick]
            %\draw[xstep=1cm, ystep=1cm, line width=0.1mm, color=lightgray] (0,0) grid (10cm,10cm);

            % Einstellungen für Symbole
            \ctikzset{
                inductor            =   american,
                inductors/scale     =   1.0,
                capacitors/scale    =   0.8,
                diodes/scale        =   0.6,
                line width          =   0.5,
            }
            \def\labelOffset{0.4}

            % Knoten
            \coordinate (input) at (0,3);
            \coordinate (P1) at (0,1);
            \coordinate (K1) at (6,3);
            \coordinate (K2) at (6,1);
            \coordinate (G) at (0,0.5);
            \coordinate (output1) at ($(K1) + (1,0)$);
            \coordinate (output2) at ($(K2) + (1,0)$);

            % Kreuzungspunkte
            \draw (input)   node[circ]{};
            \draw (K1)      node[circ]{};
            \draw (K2)      node[circ]{};
            \draw (output1)      node[circ]{};
            \draw (output2)      node[circ]{};

            % Knotenbeschriftungen
            \node at ($(input)+(0,\labelOffset)$)  {$V_{IN}$};
            \node at ($(output1)+(0,\labelOffset)$) {$V_{OUT}$};

            %Trafo%
            \draw (3,2) node[transformer, scale=0.95] (T) {}
            (T.A1) node[anchor=east] {} %A1
            (T.A2) node[anchor=east] {} %A2
            (T.B1) node[anchor=west] {} %B1
            (T.B2) node[anchor=west] {} %B2
            (T.base) node{T}
            (T.inner dot A2) node[circ]{}
            (T.inner dot B1) node[circ]{};

            %      Start Pos    Symbol type             Name        End Pos
            \draw (G)         node[tlground                    ]  {};             % GND
            \draw (0,1)         to [normal open switch,     l=$S$]  (T.A2) {};      % Switch
            \draw (T.B1)        to [diode,                  l=$D$]  (K1);          % Diode
            \draw (K1)          to [curved capacitor,       l=$C$]  (K2);           % Kondensator
            \node at ($(K2) + (0.2,1.3)$)  {+};                                     % + Symbol am Kondensator

            % Verbindungen
            \draw (input) [short] (T.A1);
            \draw (G) |- (0.1,1);
            \draw (input) -- (T.A1);
            \draw (T.B2) -- (output2);
            \draw (K1) -- (output1);

        \end{circuitikz}
    }
\end{center}

