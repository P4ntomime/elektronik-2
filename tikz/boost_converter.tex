% Aufwärtswandler (Boost, Step up)
% 
% Author:   Alex Krieg
% Date:     02.04.2024


\begin{center}
    \scalebox{0.6}{%
        \begin{circuitikz}[thick]
            % Gitternetzlinien im Hintergrund
            %\draw[xstep=1cm, ystep=1cm, line width=0.1mm, color=lightgray] (0,0) grid (6,4);

            % Einstellungen für Symbole
            \ctikzset
            {  
                inductor            =   american, 
                resistor            =   european,
                inductors/scale     =   1.0, 
                capacitors/scale    =   0.8,
                diodes/scale        =   0.6,
                line width          =   0.5,
            }
            \def\labelOffset{0.4}

            % Knoten
            \coordinate (input) at (0,2);
            \coordinate (K1) at (2,2);
            \coordinate (K2) at (4,2);
            \coordinate (output) at (5,2);
            \coordinate (G1) at (2,0);
            \coordinate (G2) at (4,0);

            % Kreuzungspunkte
            \draw (input)   node[circ]{};
            \draw (K1)      node[circ]{};
            \draw (K2)      node[circ]{};
            \draw (output)  node[circ]{};

            % Knotenbeschriftungen
            \node at ($(input)+(0,\labelOffset)$)  {$V_{IN}$};
            \node at ($(output)+(0,\labelOffset)$) {$V_{OUT}$};

            % Schaltung
            %      Start Pos    Symbol type             Name    End Pos
            \draw (input)   to [L,                      l=$L$]  (K1);               % Suple
            \draw (K1)      to [normal open switch,     l=$S$]  (G1);               % Schalter
            \draw (K1)      to [diode,                  l=$D$]  (K2);               % Diode
            \draw (K2)      to [curved capacitor,       l=$C$]  (G2);               % Kondensator
            \node at ($(G2) + (0.2,1.3)$)  {+};                                     % + Symbol am Kondensator

            \draw (K2)      to [short](output);

            \draw (G1)      node[tlground](GND){}; 
            \draw (G2)      node[tlground](GND){}; 

        \end{circuitikz}
    }
\end{center}
