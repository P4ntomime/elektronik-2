\section{Analoge Filter}

\begin{tabular}{ll@{}}
    $f_{3 \, \deci \bel}$   & Cut-Off-Frequency, Corner-Frequency \\
                            & Dämpfung von $3 \, \deci \bel$ (d.h. Amplitude wird mit $\frac{1}{\sqrt{2}}$ 'verstärkt'), Phase: $- 45 \degree$ \\
    $f_S$                   & Sampling-Frequenz (ADC, digitale Filter) \\
                            & \textrightarrow\ Alle Frequnezen über $\frac{f_S}{2}$ müssen unterdrückt werden \\
    UTF                     & Übertragungsfunktion $G(s)$
\end{tabular}


\subsection{Tiefpassfilter 1. Ordnung}

\begin{minipage}[c]{0.3\columnwidth}
    \includegraphics[width=\columnwidth]{images/tiefpass_ordnung_1.png}
\end{minipage}
\hfill
\begin{minipage}[c]{0.46\columnwidth}
    % $$ \text{UTF: } G(s) = \frac{V_{out}}{V_{in}} = \frac{\frac{1}{s \cdot C}}{R + \frac{1}{s \cdot C}} = \frac{1}{1 + s \cdot \underbrace{R \cdot C}_T} $$
    $$ \boxed{ G(s) = \frac{V_{out}}{V_{in}} = \frac{1}{1 + s \cdot \underbrace{R \cdot C}_T} } $$
\end{minipage}
\hfill
\begin{minipage}[c]{0.22\columnwidth}
    $$ \boxed{ f_{3 \, \deci \bel} = \frac{1}{2 \pi R C} } $$
\end{minipage}


\subsection{Bodeplot Tiefpassfilter 1. und 2. Ordnung}

\begin{minipage}[t]{0.48\columnwidth}
    \begin{center}
        \myul{1. Ordnung}
    \end{center}
    \begin{itemize}
        \item Abfall von $- 20 \deci \bel$ / Dekade
        \item Phasenschiebung von maximal $- 90 \degree$ (bei $f_g = -45 \degree$)
    \end{itemize}
\end{minipage}
\hfill
\begin{minipage}[t]{0.48\columnwidth}
    \begin{center}
        \myul{2. Ordnung}
    \end{center}
    \begin{itemize}
        \item Abfall von $- 40 \deci \bel$ / Dekade
        \item Phasenschiebung von maximal $- 180 \degree$ (bei $f_g = -90 \degree$)
    \end{itemize}
\end{minipage}


\subsection{Filter 2. Ordnung}

\subsubsection{Kaskadierung von zwei gleichen Filtern}

\begin{minipage}[c]{0.48\columnwidth}
    $$ G_{11}(s) = \frac{1}{1 + s \cdot \underbrace{R \cdot C}_{T_2}} \cdot \frac{1}{1 + s \cdot \underbrace{R \cdot C}_{T_2}} $$
\end{minipage}
\hfill
\begin{minipage}[c]{0.48\columnwidth}
    $$ T_2 = \frac{\sqrt{\sqrt{2} - 1}}{2 \pi f_{3 \, \deci \bel} } \approx 0.64 \cdot T_1  $$
\end{minipage}

Daraus folgt, dass bei 2 identischen Stufen die Grenzfrequenz $f_{3 \, \deci \bel}$ der einzelnen Stufen $\frac{1}{0.64} = 1.56$ mal 
\textbf{höher} gewählt werden muss als bei einem Filter 1. Ordnung.


\subsubsection{Filter 2. Ordnung mit komplexen Polen}

\begin{minipage}[c]{0.6\columnwidth}
    $$ G(s) = \frac{p_1 \cdot p_2}{(p_1 + s) \cdot (p_2 + s)} = \frac{\omega_0^2}{s^2 + \frac{s \cdot \omega_0}{Q} + \omega_0^2} $$
$$ p_{1,2} = \frac{\omega_0}{2 Q} (1 \pm \sqrt{1 - 4 Q^2}) $$
\end{minipage}
\hfill
\begin{minipage}[c]{0.38\columnwidth}
    \begin{tabular}{ll}
        $p_i$       & Polstellen \\
                    & komplex für $Q > \frac{1}{2}$ \\
        $Q$         & Polgüte / Filtergüte \\
        $\omega_0$  & Polfrequenz
    \end{tabular}
\end{minipage}



% bei Platzmangel weglassen
\subsection{Filter höherer Ordnung}

\begin{itemize}
    \item Systeme höherer Ordnung können aufgeteilt werden in kaskadierte Teilsysteme 1. und 2. Ordnung
    \item Höhere Ordnung und komplexe Pole ermöglichen steileren Übergang zwischen Durchlass- und Sperrbereich
\end{itemize}


\subsection{Zeitverhalten: Schrittantwort}

\begin{enumerate}
    \item Frenqenzbereich: \textbf{Multiplikation} der UTF mit $\frac{1}{s}$
    \item Rücktransformation in den Zeitbereich, um $t_{step}(t)$ zu erhalten
\end{enumerate}


\subsubsection{Tiefpass 1. Ordnung}

\begin{minipage}[c]{0.48\columnwidth}
    \includegraphics[width=\columnwidth]{images/schrittantwort_tp_ordnung_1.png}
\end{minipage}
\hfill
\begin{minipage}[c]{0.48\columnwidth}
    $$  t_{step,1}(t) = 1 - e^{- \frac{t}{T_1}} $$
\end{minipage}


\subsubsection{Tiefpass 2. Ordnung}

\begin{minipage}[c]{0.43\columnwidth}
    \includegraphics[width=\columnwidth]{images/schrittantworten_vergleich.png}
\end{minipage}
\hfill
\begin{minipage}[c]{0.55\columnwidth}
    $$ t_{step2a}(t) = 1 - e^{- \frac{t}{T_1}}\cdot \Big( 1 + \frac{t}{T_1} \Big) $$
    $$ t_{step2b}(t) = 1 - \Big( \frac{T_1 \cdot e^{- \frac{t}{T_1}} - T_2 \cdot e^{- \frac{t}{T_2}}}{T_1 - T_2} \Big) $$
\end{minipage}


\subsection{Schrittantworten verschiedener Polgüten}

\begin{minipage}[c]{0.48\columnwidth}
    \includegraphics[width=\columnwidth]{images/schrittantwort_verschiedene_polgueten.png}
\end{minipage}
\hfill
\begin{minipage}[c]{0.48\columnwidth}
    Bei einer Polgüte von $Q = \frac{1}{\sqrt{2}} \approx 0.7$ (\cgn{grüne Kruve}) schwingt das System am schnellsten ein!
\end{minipage}