\section{Anhang}


\subsection{E-Reihen}

\scalebox{0.58}{

\begin{tabular}{| c|c|c|c | c|c|c|c | c|c|c|c | c|c|c|c | c|c|c|c | c|c|c|c |}
    \toprule
    \cbl{10}    &       & 12    &       & \cbl{15}  &       & 18    &       & \cbl{22}  &       & 27    &       & \cbl{33}  &    & 39   &       & \cbl{47}  &       & 56    &       & \cbl{68}  &       & 82    &       \\
    \midrule
                & 11    &       & 13    &           & 16    &       & 20    &           & 24    &       & 30    &           & 36 &      & 43    &           & 51    &       & 62    &           & 75    &       & 91    \\
    \bottomrule
\end{tabular}
}

\begin{tabular}{ll c ll}
    E6-Reihe:   & \cbl{blau markierte}  & & E12-Reihe: & obere Zeile \\
    E24-Reihe:  & ganze Tabelle
\end{tabular}


\subsection{Elektrische Eigenschaften typischer Materialien}

\begin{tabular}{ll cc ll}
    $\rho$  & Spezifischer Widerstand & & & $\alpha$    & Temperaturkoeffizient \\
\end{tabular}

\begin{center}
    \begin{tabular}{ l | l  c  c }
        \toprule
        \textbf{Typ}        & \textbf{Material}     & $\bm{\rho}$ \textbf{in} $\bm{\frac{\ohm \cdot \milli \meter^2}{\meter}}$  & $\bm{\alpha}$ \textbf{in} $\bm{\frac{1}{\kelvin}}$    \\
        \midrule
        Leiter              & Silber                & $0.016$                                                                   & $3.8 \cdot 10^{-3}$                                   \\
        \midrule
                            & Kupfer                & $0.0178$                                                                  & $3.92 \cdot 10^{-3}$                                  \\
        \midrule            
                            & Gold                  & $0.023$                                                                   & $4 \cdot 10^{-3}$                                     \\
        \midrule            
                            & Aluminium             & $0.028$                                                                   & $3.77 \cdot 10^{-3}$                                  \\
        \midrule
                            & Konstantan            & $0.43$                                                                    & $\pm 40 \cdot 10^{-6}$                                \\  
        \midrule
        Halbleiter          & Silizium              & $6.25 \cdot 10^6$                                                         & $- 1 \cdot 10^{-3}$                                   \\
        \midrule
                            & Germanium             & $0.454 \cdot 10^6$                                                        & $- 5 \cdot 10^{-3}$                                   \\
        \midrule        
        Isolator            & Porzellan             & $5 \cdot 10^{18}$                                                         &                                                       \\
        \bottomrule               
    \end{tabular}
\end{center}


\subsection{Dielektrizitätskonstanten einiger Materialien}

\begin{center}
    \begin{tabular}{l | l }
        \textbf{Material}       &  $\bm{\varepsilon_r}$ \\
        \midrule
        Vakuum, Luft            & $1$                   \\
        FR4                     & $\approx 4.7$         \\
        Glas                    & $5-10$                \\
        Aluminium Oxid (Elko)   & $10$                  \\
        Tantal Oxid (Elko)      & $26$                  \\
        Keramik                 & $10 - 20'000$         \\
        Wasser                  & $80$  
    \end{tabular}
\end{center}