\section{Signalübertragung}

\subsection{Leitungstheorie}

\begin{itemize}
    \item Leitungen haben Widerstände, Kapazitäten und Induktivitäten \textrightarrow\ RLC-Netzwerke
    \item \textbf{Fortpflanzungsgeschwindigkeit Signal:} $v = 10 - 20 \, \centi \meter / \nano \second$ \\
        (Lichtgeschwindigkeit: $c = 0 \, \centi \meter / \nano \second$) % TODO: 0 correct?
    \item Ev. \textbf{Impedanzanpassungen} zur Verhinderung von \textbf{Reflexionen} nötig (meistens $50 \, \ohm$)
    \item CMOS-Logik: tiefen Quellenwiderstand, hohen Eingangswiderstand \\
        \textrightarrow\ Nicht geeignet zur Datenübertragung über 'längere Strecken'
\end{itemize}


\subsection{Einfluss / Relevanz von Refelxionen}

\subsubsection{Keine Reflexionen}

Wenn nichts anderes bekannt gilt: $T_r = \frac{1}{10} \cdot T$ 

\begin{minipage}[c]{0.3\columnwidth}
    $$ \boxed{ T_d < \frac{1}{2} \cdot T_r} $$
\end{minipage}
\hfill
\begin{minipage}[c]{0.68\columnwidth}
    \begin{tabular}{ll}
        $T_r = T_f$ & Anstiegs- / bzw. Abfallzeit des Signals \\
        $T_d$       & Laufzeit des Signals \\
        $T$         & Periodendauer
    \end{tabular}
\end{minipage}


\subsubsection{Reflexionen}

\begin{minipage}[c]{0.3\columnwidth}
    $$ \boxed{ l > \frac{1 \cdot 10^7 \, \frac{\meter}{\second} }{f_{\rm max}} } $$
\end{minipage}\hfill
\begin{minipage}[c]{0.68\columnwidth}
    \begin{tabular}{ll}
        $f_{\rm max}$   & Maximal enthaltene Frequenz im Signal \\
        $l$         & Länge der Leitung 
    \end{tabular}
\end{minipage}

