\section{Schmitt-Trigger}

\begin{itemize}
    \item Schaltschwellen müssen nicht sehr genau sein
    \item Schmitt-Trigger garantieren auch bei verrauschten Signalen saubere (einmalige) Schaltschwellen, dank der Hysterese
\end{itemize}



\subsection{Aufbau nichtinvertierender digitaler Schmitt-Trigger}

\begin{minipage}[position]{0.35\columnwidth}
    \includegraphics[width=\columnwidth]{images/nichtinvertierender_schmitt-trigger.png}
\end{minipage}
\hfill
\begin{minipage}[position]{0.63\columnwidth}
    \begin{itemize}
        \item $M_1, M_2$: Digitale Inverter
        \item $M_3, M_4$: 'gesteuerte Widerstände
        \item \textbf{Für }$\boldsymbol{V_{out} = 0}$: $M_4$ leitet, $M_3$ sperrt
        \item \textbf{Für }$\boldsymbol{V_{out} = 1}$: $M_3$ leitet, $M_4$ sperrt
        \item $M_3, M_4$ verschieben Schaltschwellen abhängig von $V_{out}$ \textrightarrow Hysterese
    \end{itemize}
\end{minipage}


\subsection{Aufbau invertierender digitaler Schmitt-Trigger}

\begin{minipage}[position]{0.3\columnwidth}
    \includegraphics[width=\columnwidth]{images/invertierender_schmitt-trigger.png}
\end{minipage}
\hfill
\begin{minipage}[position]{0.68\columnwidth}
    \begin{itemize}
        \item Ohne $M_5, M_6$: Normaler Inverter mit je 2 Serie-Transistoren
        \item \textbf{Für }$\boldsymbol{V_{out} = 1}$: Durch $M_5$ fliesst Strom in $M_1$
        \item $V_{in}$ muss höher sein, um Strom der PMOS aufzunehmen \textrightarrow Höhere Schaltschwelle für High-Log-Übergang
        \item 'Inverses' gilt für $M_6$ und $M_4$
    \end{itemize}
\end{minipage}


\subsection{Schmitt-Trigger vs. CMOS-Logik}

\includegraphics[width=\columnwidth]{images/benefits_schmitt-trigger.png}